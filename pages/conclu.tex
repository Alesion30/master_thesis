%#! platex main.tex

%======================================================================
\chapter{おわりに}
\label{cha:conclu}

本稿では, 食事中の音を録音と咀嚼回数のカウントを同時に行うアプリケーションの設計・開発を行い, 食事中の音から16種類の食事内容を推定する手法および10秒間の食事中の音から咀嚼回数を検出する手法を新たに提案した. これらの分析するために, 研究室の学生15名の被験者を対象に, AirPodsを装着した上で食事を行ってもらい, 合計13422秒の食事中の音を収集した. また音データをスライディングウィンドウで分割し, 分割後の音データからメルスペクトログラムを算出し, 畳み込みニューラルネットワークを学習させ, 食品認識モデルを作成した. その結果, 検証データに対して精度$77.5\%$を確認することができた. また, 食事中の音をメルスペクトログラムに変換し, 各時間軸毎の全ての周波数帯の信号強度の平均を取り, ピーク検出を行うことで, 咀嚼回数の検出を行なった. 10秒間の音データに対して被験者がカウントした咀嚼回数とピーク検出回数との間の平均絶対値誤差$MAE$を算出したところ, $MAE = 4.9$を確認することができた. 今回はノイズの少ない静かな環境で計測したため, ノイズのある環境での精度向上を目指す. また, リアルタイムで食事内容や咀嚼回数の推定を行うことで, 個人の食事の順番やスピードを記録し, 食事行動改善に向けたアプリケーションの実現を目指す.


% 以下はRefTeX用
%%% Local Variables:
%%% mode: yatex
%%% TeX-master: "main"
%%% End:
