%#! platex main.tex

%======================================================================
\chapter{関連研究}
\label{cha:research}


% 本章では, 食品認識・咀嚼検出に関する関連研究について述べた後に, 食事行動改善に向けた研究の現状と課題を整理する. さらに本研究の手法であるイヤラブルデバイスを用いた人間活動認識(HAR)について言及し, 本研究の立ち位置を述べる.

本章では, 食品認識・咀嚼検出に関する関連研究について述べた後に, 本研究の手法であるイヤラブルデバイスを用いた人間活動認識(HAR)について述べる.

\begin{table*}[t]
    \centering
    \caption{食品認識に関する関連研究}
    \label{food_recognition_related_work}
    \scalebox{0.85}{
        \begin{tabular}{c|c|c|c}
            \hline
            \textbf{年度} & \textbf{参考文献}                  & \textbf{手法}                                                           & \textbf{使用するデバイス}                                    \\ \hline\hline
            2016        & \cite{7520547}                 & 単一の食品画像を分類(画像ベース)                                                     & カメラ                                                  \\ \hline
            2017        & \cite{10.1145/3063592}         & 複数の食品画像を分類(画像ベース)                                                     & カメラ                                                  \\ \hline
            2023        & \cite{asi6020053}              & \begin{tabular}{c}リアルタイムで食品認識を行う\\モバイルアプリケーションを提案(画像ベース)\end{tabular}
                        & スマートフォン                                                                                                                                                       \\ \hline

            2022        & \cite{10.1145/3551626.3564964} & \begin{tabular}{c}食事中の一人称映像から時系列毎に検出\\(動画ベース)\end{tabular}            & カメラ                                                  \\ \hline

            2022        & \cite{app12126135}             & \begin{tabular}{c}食事中の気管下部から皮膚の動きから推定\\(センサベース)\end{tabular}          & \begin{tabular}{c}独自のネックレス型の\\ウェアラブルセンサ\end{tabular} \\ \hline

            2023        & 提案手法                           & 食事中に発生する音から推定                                                         & \begin{tabular}{c}ワイヤレスイヤホン\end{tabular}             \\ \hline
        \end{tabular}
    }
\end{table*}

%----------------------------------------------------------------------
\section{食品認識}

食品認識とは, コンピュータビジョンと機械学習の技術を使用して, 画像やビデオ内の食品を識別し, 分離するプロセスのことである. 本節では, 食品認識に関する関連研究を, 画像ベース, 動画ベース, センサベースの3つに分類して述べる. 食品認識に関する関連研究を表\ref{food_recognition_related_work}に示す.

%----------------------------------------------------------------------
\subsection{画像ベース}

画像を利用した分析手法は, 食品認識において最も基本的な手法で, 近年では特にディープラーニングによる画像認識手法が主流となりつつある\cite{10.1145/3063592}\cite{10.1145/3063592}\cite{asi6020053}. また, カロミルやあすけん\footnote{あすけん | あすけんダイエット - 栄養士が無料であなたのダイエットをサポート~\url{https://www.asken.jp/}}などをはじめとした食事管理アプリケーションにもすでに広く統合されており, 食事メニューの推定だけでなく, そこから摂取栄養素を記録することができる. その他の応用例として, パンの画像から種類を識別するパン画像認識レジ「BakeryScan」\footnote{BakeryScan(ベーカリースキャン)~\url{https://bakeryscan.com/}}というものまで存在する. しかし, 画像は非時系列データなので, リアルタイムでの食事内容の推定には向かないという問題点が存在する.

%----------------------------------------------------------------------
\subsection{動画ベース}

動画を利用した分析手法は, 先ほど取り上げた画像ベースの手法の応用例となっており, ウェアラブルカメラで撮影した食事中の一人称映像をフレームに分割することで, 時系列順に食事内容の推定を行うことができる\cite{10.1145/3551626.3564964}. しかし, 食事の様子を常にカメラで撮影し続ける必要があるため, 一般的な食事シーンにおいては受け入れ難いという問題点が存在する.

%----------------------------------------------------------------------
\subsection{センサベース}

センサデータは時系列データであるため, 食事内容をリアルタイムで検出することができる. 独自のウェアラブルデバイスを用いた手法として, 食事中の気管下部から皮膚の動きを検出する圧電センサを組み込んだネックレス型のウェアラブルセンサが存在する\cite{app12126135}. しかしながら, 独自のウェアラブルデバイスの形状によっては, 一般的な食事シーンにおいて着用が難しく, そもそも独自のウェアラブルデバイスを普及させる必要がある.

%----------------------------------------------------------------------
\section{咀嚼検出}

咀嚼とは, 歯を使って食物を細かく砕くことであり, 咀嚼データは個人の摂食パターンや食生活を理解するために利用することができる. 本節では, Selamatらによってまとめられた咀嚼検出ベースの摂食モニタリングの調査\cite{9024026}を元に, さらに最新研究を交えつつ, 咀嚼検出に関する関連研究を, 音響ベース・運動ベースに分類して述べる. 咀嚼検出に関する関連研究を表\ref{chewing_detection_related_work}に示す.

\begin{table*}[t]
    \centering
    \caption{咀嚼検出に関する関連研究}
    \label{chewing_detection_related_work}
    \scalebox{0.85}{
        \begin{tabular}{c|c|c|c}
            \hline
            \textbf{年度} & \textbf{参考文献}  & \textbf{手法}            & \textbf{使用するデバイス}                        \\ \hline\hline
            2010        & \cite{5690449} & スペクトル音分析による検出(音響ベース)   & ウェアラブルイヤーパッド                             \\ \hline
            2017        & \cite{8037060} & 外耳に装着したマイクから検出(音響ベース)  & イヤラブルマイク                                 \\ \hline
            2012        & \cite{6047558} & 圧電センサで顎の動きを検出(運動ベース)   & ピエゾ式歪みゲージセンサ                             \\ \hline
            2022        & \cite{9942809} & 近接センサで側頭筋の動きを検出(運動ベース) & 近接センサ                                    \\ \hline

            2023        & 提案手法           & 食事中に発生する音から推定          & \begin{tabular}{c}ワイヤレスイヤホン\end{tabular} \\ \hline
        \end{tabular}
    }
\end{table*}

%----------------------------------------------------------------------
\subsection{音響ベース}

音響センシングでは, マイクを用いて咀嚼音を捉えることで, 咀嚼を検出する. この手法は, 最も古くから研究されている手法の一つであり, 現在でも多く活用されている.\cite{5690449}\cite{8037060} しかし, 音響センサには環境音などのノイズの影響を受けやすいという問題点が存在する.

%----------------------------------------------------------------------
\subsection{運動ベース}

運動センシングでは, 例えば顎の動きといった身体の一部分の動きを検出することで, 咀嚼を検出する.\cite{6047558}\cite{9942809} この手法は, 音響センシングと比較して, 環境音の影響を受けにくいという利点がある. しかし, 顎の動きを検出するためには, 身体にセンサを装着する必要があり, センサの形状によっては使用者に不快感を与えたり, センサ精度が歩行や会話といった他の日常動作に影響される可能性があるという問題点が存在する.

% %----------------------------------------------------------------------
% \section{食事行動改善に向けた研究の現状と課題}

%----------------------------------------------------------------------
\section{イヤラブルデバイスを用いたHAR}

近年, イヤラブルなデバイスを用いたHuman Activity Recognition(HAR)に関する研究が活発に行われてきている. 本章では, 一般的なハイエンドなワイヤレスイヤホンに搭載されている慣性計測ユニット(IMU)を用いた手法と従来のワイヤレスイヤホンにも搭載されているマイクを用いた手法について述べる.

%----------------------------------------------------------------------
\subsection{IMUベース}

IMUを搭載したウェアラブルデバイスを耳に装着することで, 特に頭の動きを認識することができる. Tahera HossainらはIMUを搭載したワイヤレスイヤホンを用いて, 頭や口に関連する行動である6つの活動(話す・食べる・首を振る・頷く・とどまる・歩く)を分類する手法を提案している\cite{10.1145/3341162.3343822}. また, Dhruv VermaらもIMUを搭載した市販のワイヤレスイヤホンを用いて, 46種類の表情を認識するExpressEarというシステムを提案している\cite{10.1145/3478085}.

%----------------------------------------------------------------------
\subsection{マイクベース}

マイクを用いた手法は, 非常に多種多様な研究が存在する. Yuntao Wangらは, イヤホンを用いた音響測距に基づく新しい顔追跡技術であるFaceOriを提案している\cite{10.1145/3491102.3517698}. また, Xuhai Xuらは, 市販のワイヤレスイヤホンのマイクから顔周りのジェスチャを検出し, さらにジェスチャからスマートフォンを操作できるEarBuddyというシステムを提案している\cite{10.1145/3313831.3376836}. また, 歯のジェスチャを行う際に発生する音波をユーザー認証に活用するといった変わった研究も存在する\cite{10.1145/3460120.3485340}.

イヤラブルデバイスのマイクは, 特に口から発生する音を捕捉することができるため, 本研究でもこのアプローチを採用する.

%%% Local Variables:
%%% mode: yatex
%%% TeX-master: "main"
%%% End:
