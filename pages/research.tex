%#! platex main.tex

%======================================================================
\chapter{関連研究}
\label{cha:research}

本章では, 食品認識に関する関連研究\ref{related_work}について述べたあと, 本研究の手法であるイヤラブルデバイスを用いたHARについて言及する.

\begin{table*}[t]
    \centering
    \caption{食品認識に関する関連研究}
    \label{related_work}
    \scalebox{0.9}{
        \begin{tabular}{c|c|c|c}
            \hline
            \textbf{年度} & \textbf{参考文献}                  & \textbf{手法}                                                  & \textbf{使用するデバイス}                                    \\ \hline\hline
            2017        & \cite{10.1145/3063592}         & 食事の画像から推定                                                    & カメラ(スマートフォン)                                         \\ \hline
            2019        & \cite{1523669555317207552}     & \begin{tabular}{c}パンの画像から種類を識別し, \\パン専用のレジとして応用\end{tabular} & 独自のレジ装置                                              \\ \hline
            2022        & \cite{10.1145/3551626.3564964} & 食事中の一人称映像から時系列毎に検出                                           & カメラ                                                  \\ \hline
            2022        & \cite{app12126135}             & 食事中の気管下部から皮膚の動きから推定                                          & \begin{tabular}{c}独自のネックレス型の\\ウェアラブルセンサ\end{tabular} \\ \hline
            2023        & 提案手法                           & 食事中に発生する音から推定                                                & \begin{tabular}{c}マイクを搭載した\\イヤラブルデバイス \end{tabular}  \\ \hline
        \end{tabular}
    }
\end{table*}

%----------------------------------------------------------------------
\section{食品認識}

食品認識における関連研究の中で, (1)食事の画像から食品を推定する手法, (2)ユーザ自身の食事行動を撮影した一人称の動画から時系列順で食品推定を行う手法, (3)独自のウェアラブルデバイスを用いて計測したセンサデータを元に食品推定を行う手法について述べる.

%----------------------------------------------------------------------
\subsection{画像ベース}

画像を利用した分析手法は, 食品認識において最も基本的な手法で, 近年では特にディープラーニングによる画像認識手法が主流となりつつある\cite{10.1145/3063592}. また, 実際の食事管理アプリケーションにもすでに広く統合されており, 食事メニューの推定だけでなく, そこから摂取栄養素を記録することができる. その他にも応用例として, パンの画像から種類を識別するパン画像認識レジ「BakeryScan」というものまで存在する\cite{1523669555317207552}. しかし, 画像は非時系列データなので, リアルタイムでの食事内容の推定には向かないという問題点が存在する.

%----------------------------------------------------------------------
\subsection{動画ベース}

動画を利用した分析手法は, 先ほど取り上げた画像ベースの手法の応用例となっており, ウェアラブルカメラで撮影した食事中の一人称映像をフレームに分割することで, 時系列順に食事内容の推定を行うことができる\cite{10.1145/3551626.3564964}. しかし, 食事の様子を常にカメラで撮影し続ける必要があるため, 一般的な食事シーンにおいては受け入れ難いという問題点が存在する.

%----------------------------------------------------------------------
\subsection{センサベース}

センサデータは時系列データであるため, 食事内容をリアルタイムで検出することができる. 独自のウェアラブルデバイスを用いた手法として, 食事中の気管下部から皮膚の動きを検出する圧電センサを組み込んだネックレス型のウェアラブルセンサが存在する\cite{app12126135}. しかしながら, 独自のウェアラブルデバイスの形状によっては, 一般的な食事シーンにおいて着用が難しく, そもそも独自のウェアラブルデバイスを普及させる必要がある.

%----------------------------------------------------------------------
\section{イヤラブルデバイスを用いたHAR}

近年, イヤラブルなデバイスを用いたHuman Activity Recognition(HAR)に関する研究が活発に行われてきている. 本章では, 一般的なハイエンドなワイヤレスイヤホンに搭載されている慣性計測ユニット(IMU)を用いた手法と従来のワイヤレスイヤホンにも搭載されているマイクを用いた手法について述べる.

%----------------------------------------------------------------------
\subsection{IMUベース}

IMUを搭載したウェアラブルデバイスを耳に装着することで, 特に頭の動きを認識することができる. Tahera HossainらはIMUを搭載したワイヤレスイヤホンを用いて, 頭や口に関連する行動である6つの活動(話す・食べる・首を振る・頷く・とどまる・歩く)を分類する手法を提案している\cite{10.1145/3341162.3343822}. また, Dhruv VermaらもIMUを搭載した市販のワイヤレスイヤホンを用いて, 46種類の表情を認識するExpressEarというシステムを提案している\cite{10.1145/3478085}.

%----------------------------------------------------------------------
\subsection{マイクベース}

マイクを用いた手法は, 非常に多種多様な研究が存在する. Yuntao Wangらは, イヤホンを用いた音響測距に基づく新しい顔追跡技術であるFaceOriを提案している\cite{10.1145/3491102.3517698}. また, Xuhai Xuらは, 市販のワイヤレスイヤホンのマイクから顔周りのジェスチャを検出し, さらにジェスチャからスマートフォンを操作できるEarBuddyというシステムを提案している\cite{10.1145/3313831.3376836}. また, 歯のジェスチャを行う際に発生する音波をユーザー認証に活用するといった変わった研究も存在する\cite{10.1145/3460120.3485340}.

イヤラブルデバイスのマイクは, 特に口から発生する音を捕捉することができるため, 本研究でもこのアプローチを採用する.

%%% Local Variables:
%%% mode: yatex
%%% TeX-master: "main"
%%% End:
